\addtolength{\oddsidemargin}{-.875in}
\addtolength{\evensidemargin}{-.875in}
\addtolength{\textwidth}{1.75in}

\addtolength{\topmargin}{-.875in}
\addtolength{\textheight}{1.75in}
\title{PGM Notes}
\author{
   Amit Indap
}
\date{\today}

\documentclass[12pt]{article}

\begin{document}
\maketitle

\begin{abstract}
\end{abstract}
%%%%%%%%%%%%%%%%%%%%%%%%%%%%%%%%
\section{Introduction}
%%%%%%%%%%%%%%%%%%%%%%%%%%%%%%%%

\subsection{Joint Distribution and CPDs}
Say you are given a list of n random variables $X_1  \ldots X_N$ and for simplicity�s sake each of these are binary valued. The join distribution, P , of these n variables would require the specification of 2n ? 1 numbers.
 Save for the smallest values of n,  the joint distribution is cumbersome to represent and manipulate computationally. To represent joint distributions more compactly Bayesian Networks (BNs) take advantage of independence 
 properties of variables and use the directed acyclic graph (DAG) as a general purpose data structure to encode, manipulate, and calculate probabilities of high dimensional distributions. \\
 
 \noindent Taking the example from Section 3.1.2 in  [1], consider a company trying to higher recent graduates. They want to hire gifted employees, but there is no direct way to assess intelligence. 
 Rather, they use a student�s SAT score as a proxy. We can formalize this probabilistically by two random variables, Intelligence (I) and SAT (S). To represent the joint distribution of S and I, we can factorize the joint in the following way: 
 $P(I,S) = P(I)P(S|I)$. Mathematically, P(I) represents the prior distribution over I while P(S|I is a conditional probability distribution (CPD) of S given I. 
 Figure 1 shows how this can be shown as a DAG,  representing each of the random variables as nodes and edges representing dependencies between variables. 


\subsection{Factors}

\section{Representing factors in Python/Matlab}
Representing graphical models and performing inference is done by creating and manipulating factors. A factor is function É.. \\

\noindent The Coursera class used Matlab/Octave for its programming assignments. As far as I know there is now Matlab library to do I/0 on BAM files. Therefore, I needed to implement
factor representation in Python, using the PGM Matlab code as a template. 

  
%%%%%%%%%%%%%%%%%%%%%%%%%%%%%%%%
\section{Variable Elimination and Exact Inference}     \label{VE} %                    
%%%%%%%%%%%%%%%%%%%%%%%%%%%%%%%%
The common feature of any inference techniques with PGMs are the manipulation of factors. A factor $\phi$ over a $Scope[\phi] = X$, which is a function $\phi:Val(X) \mapsto R$
The underlying operation when computing the probability of some variable in a PGM is marginalizing out variables from a distribution. We can view the as computation on a factor. Let $X$ be a set of variables and $Y \notin X$ and $\psi(X,Y)$ be a factor. Marginalizing out $Y$ generates a new factor $\psi$ over $X$:

\begin{eqnarray}
\psi(X) = \sum_X \phi(X,Y)
\end{eqnarray} \label{marginalize}

\noindent A key trick in doing inference on PGMs is exchanging a summation and a product if $X \notin Scope[\phi_1]$:

\begin{eqnarray}
\sum_X (\phi_1 \phi_2) = \phi_1 \sum_X(\phi_2)
\end{eqnarray}

\noindent A marginal probability computation involves taking the product of CPDs and doing a summation on all the variables except the query variables (the variables you want to find the posterior for). So in general, the inference task involves taking  a \textit{sum-product} of the form

\begin{eqnarray}
\sum_Z \prod_{\phi \in \Phi} \phi
\end{eqnarray} \label{sumprod}

\noindent The sum product variable elimination (VE) pseudo code is given on page 298. Each variable is summed out one at a time, given particular elimination order. When a variable is summed out, all factors that contain that variable in its scope are multiplied, generating a product factor. Then the variable to be eliminated is summed out of this product factor. 
Again, let $X$ be a set of variables and $\Phi$ be a set of factors such that for each $\phi \in \Phi$, $Scope[\phi] \subseteq X$ Let $Y \subset X$ be a set of query variables and the remaining variables be $Z=X-Y$. Then for any elimination ordering of the set of variables, then the sum product variable elimination returns a new factor $\phi^*(Y)$:

\begin{eqnarray}
\phi^*(Y) =\sum_Z \prod_{\phi \in \Phi} \phi
\end{eqnarray} \label{ve_sumprod}
%%%%%%%%%%%%%%%%%%%%%%%%%%%%%%%%
\subsection{ Graph theoretic view of Variable Elimination}         %
%%%%%%%%%%%%%%%%%%%%%%%%%%%%%%%%
The sum product VE algorithm is agnostic about the type of graph on which it operates. But the manipulation of factors can equivalently be seen as a series of graph transformations with an associated set of factors. Define an undirected graph $H$ whose nodes are variables in the $Scope[\Phi]$ and where there is an edge between nodes if there exists a factor $\phi \in \Phi$ such that $X_i$ and $X_j \in Scope[\phi]$ In other words, the undirected graph $H_\Phi$ is a fully connected sub-graph over the scope of each factor $\phi \in \Phi$. In the case of no evidence the graph $H_\Phi$ is a moralized graph of the Bayesian Network $G$. \\

\noindent  In the process of eliminating a variable a new factor $\psi$ is created with X and all the other variables $\textbf{Y}$ that appear with it in factors. Then $X$ is summed out, creating a new factor $\tau$ that contains all the variables \textbf{Y} but not $X$. Let $\Phi_X$ be the resulting set of factors.  When the factor $\psi$ is creates edges between all the variables $Y \in \textbf{Y}$. Some may have been in the original graph $H_\Phi$, others are introduced as fill edges. When the factor $\tau$ is created, X is removed and  all its incident edges are removed. The elimination order than reflects a series of graphs and every factor that appears in the steps of the VE sum product algorithm is a clique. The set of factors generated in VE is a clique in the \textit{induced} graph. Figure 9.11 on page 309 shows the induced graph and the resulting clique tree for the VE algorithm.  

%%%%%%%%%%%%%%%%%%%%%%%%%%%%%%%%
% NEW SECTION
%%%%%%%%%%%%%%%%%%%%%%%%%%%%%%%%
\section{Clique Trees and Exact Inference}                                     %
%%%%%%%%%%%%%%%%%%%%%%%%%%%%%%%%

In section \ref{VE} we describe the VE algorithm which sums out variables one at a time. In Chapter 10, Koller and Friedman describe how to use a cluster graph/ clique tree as a global data structure to eliminate larger sets of variables.

\subsection{Variable Elimination and Clique Trees}
Recall again in each step in VE a factor $\psi_i$ is created by multiplying together factors and a variable is eliminated from $\psi_i$ to create a  new factor $\tau_i$. This process is continued till the algorithm is finished. The generation of factors can be seen as \textit{message passing} where a factor $\psi_i$ takes incoming message $\tau_j$ generated by factors $\psi_j$, then generates its own message $\tau_i$ which in turn is passed onto another factor $\psi_l$.  Figure 10.1 on page 346 shows a \textit{cluster graph} , which is a data structure for message passing. Each node in the cluster graph are a set of variables and whose edges have variable scopes with a non-empty intersection. The cluster graph is a more general data structure than a \textit{clique tree}, defined later in chapter 10, but the cluster graph is again discussed in chapter 11. \\ 

\noindent A cluster graph $U$ is a for a set of factors $\Phi$ over $X$ is an undirected graph whose nodes are associated with a subset $C_i \subseteq X$ The cluster graph
must be \textit{family preserving}  such that each factor $\phi \in \Phi$ is associated with a cluster $C_i$, denoted as $\alpha(\phi)$,  such that the $Scope[\phi] \subseteq C_i$
Each edge between clusters $C_i$ and $C_j$ is associated with the sepset $S_{i,j} \subseteq C_i \cap C_j$.  \\

\noindent \textit{An execution of variable elimination defines a cluster graph}  Compare figure 10.1 on page 346 to the elimination  process in table 9.1 on page 302. 
When an intermediate factor $\tau_i$ is generated, its used at most only once: $\phi_i$ is used to create $\psi_j$, its removed front the set of factors $\Phi$, and cannot be used again. The cluster graph induced by VE is a tree. The tree defined by an execution of variable elimination satisfies the \textit{running intersection property (RIP)}:
Let $ T $ be a cluster tree over a set of factors $ \Psi $. $T$ has the RIP whenever there is a variable $X$ such that $X \in C_i$ and $X \in C_j$ then $X$ is in every cluster in the unique path between $T$ between $C_i$ and $C_j$ 
 Now when in a cluster tree induced by a variable elimination algorithm over a set of factors $\Phi$, a cluster $C_i$ passes a message $\tau_i$ to cluster $C_j$, 
the scope of the message is the intersection between $C_i$ and $C_j: C_i \cap C_j$. If cluster  tree over $\Phi$ that satisfies the RIP property is called a \textit{clique tree} (also known as a junction tree or join tree). In a clique tree the clusters are called cliques. 

%%%%%%%%%%%%%%%%%%%%%%%%%%%%%%%%
\subsection{ Sum Product Message Passing}                                %
%%%%%%%%%%%%%%%%%%%%%%%%%%%%%%%%
An execution of VE results in a clique tree. But you can start with a clique tree and use it as a data structure to perform variable elimination. 
The same clique tree can be used multiple times for different executions of VE. So given a tree that satisfies family preservation and the RIP property, you can do can use it 
in several different ways to do inference with graphical models.  Using the CT, each clique takes incoming messages and passes outgoing messages to another clique. \\

\noindent Step 1: Calculate initial potentials by multiplying factors assigned to a clique \\

\noindent  A clique can only send messages once its received all its incoming messages and multiplied it with its own initial potential. In the clique tree message passing algorithm, a node in the CT is selected as a root. Then for each of the other nodes, define its neighbors on the path to the root. Each clique performs its message passing 
computation and sends its message to an upstream neighbor. The passing proceeds up the tree culminating at the root. When the root gets all its messages, it multiples them by its initial potential producing the final belief. Then from the final belief you can extract out your query nodes by summing out variables from the final belief factor. \\

%%%%%%%%%%%%%%%%%%%%%%%%%%%%%%%%
\subsection {Clique Tree Calibration}                                                %
%%%%%%%%%%%%%%%%%%%%%%%%%%%%%%%%
The clique tree MP algorithm can compute the probability of any variable in $X$. But what if you have several variables you want measure? You can do inference on each variable separately.  Or you can run the algorithm making each node in the CT the root. But there is a better way! A technique called \textit{sum-product belief propagation} 
can asynchronously perform inference. Given a CT , $T$, a clique $C_i$ is ready to transmit its message to a neighbor $C_j$ once $C_i$ has all its messages from its neighbors except $C_j$ BP on the CT consists of one upward pass by picking a root and passing a message up to the root. When this finished, it can send messages to all its children  ( till all the leaves are reached). This second part is the downward pass.  The root then is the first clique that gets all its messages from its neighbors. At the end of the algorithm, you can calculate final beliefs for \textit{all} cliques in the tree.  The cost of this inference is $2c$ versus $Kc$ where $K$ is the totla number cliques. \\

\noindent Now at the end of this 2 pass calibration, if a variable $X$ appears in two cliques, the cliques should agree on the marginal probability of the variable. A CT is calibrated if fore every pair of neighbors, the cliques are agree on septet beliefs. On page 357 there is algorithm pseudo-code for calibrating a clique tree (i.e. sum-product belief propogation). The main advantage of  sum-product clique tree  calibration algorithm is it computes the posterior probability of all variables using only twice the computation of the upward pass of the same tree. For a clique tree, there are two edges associated with it: incoming and outgoing. So if there are $c$ cliques, there are then $c-1$ edges in the CT, so we have $2(c-1)$ edges to calculated. Now if if we doing a separate calculation for each variable $2c$ and $Kc$ if we treat each of the $K$ cliques as the root. In general, the clique tree algorithm is the best way to calculate posterior probability of multiple query variables. 
%\bibliographystyle{abbrv}
%\bibliography{main}

\end{document}
This is never printed
